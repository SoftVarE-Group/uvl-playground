\documentclass[
	aspectratio=169, % default is 43
	8pt, % font size, default is 11pt
	%handout, % handout mode without animations, comment out to add animations
	%nosectionframes, % disable automatic frames at the begin of each section (default: sectiontitleslides in beamer mode and sectionoverviews in handout mode)
	%sectiontitleslides, % enable an automatic section title slide at the begin of each section
	%sectionoverviews, % enable an automatic section overview at the begin of each section
	%uniqueslidenumber, % will uniquely identify pages with overlays by a little suffix
	%darkmode, % switch to dark mode
]{beamer}

\usepackage{../beamerthemeuulm} % use the inofficial uulm beamer theme
\setfaculty{infIngPsy} % set the color scheme for your faculty here [med/infIngPsy/math/nat]

%\institutelogo{sp} % set the institute logo
%\universitylogo{uulm} % set a new university logo
%\clearuniversitylogo % clear existing university logo
%\clearinstitutelogo % clear existing institute logo
%\uulmlogos{sp,uulm} % freely configure multiple logos (overwrites any other logo setting)
%\uulmlogos{softvare,sp,uulm} % include softvare working group logo

%\usepackage[ngerman]{babel} % use this line for slides in German

%\setmycolumnsdefault{keep} % change the default for 'mycolumns' environment (e.g., 'keep' to animate all column environments per default)

%\includeonlyframes{current} % default mechanism of beamer to include only labeled frames, can be used for debugging or drafting

\title[UVL Playground]{UVL Playground} % short title is used for the slide footer but optional
\subtitle[Final Presentation]{Final Presentation} % subtitles are optional at all
\author[Dommer, Vill]{Jannis Dommer, Stefan Vill} % short author title is used for the slide footer but optional
\date{10.01.2024} % use a particular date here if needed

\begin{document}

\maketitle % title page with default picture

\begin{frame}{Motivation}
	\centering
	\pic[width=0.735\textwidth]{../pics/screenshots/playgrounds.png}

\end{frame}

\begin{frame}{Motivation}
	\centering
	\pic[width=0.5\textwidth]{../pics/screenshots/uvls_ts.png}
\end{frame}

\begin{frame}{Architektur}
	\begin{mycolumns}[columns=2,t]
		\pic[width=\textwidth]{../pics/screenshots/option1_final.png}
		\mynextcolumn
		\pic[width=\textwidth]{../pics/screenshots/option2_final_fertig_2.png}
	\end{mycolumns}
\end{frame}

\begin{frame}{Probleme mit WebAssembly}
	\begin{itemize}
		\item Wenig WebAssembly kompatible dependencies
		\item Systemaufrufe
		\item UVLS enthält Webserver für config
		\item Treesitter mit C-Schnittstelle
		\item Z3 mit stdin / stdout Schnittstelle
		\item Module stark gekoppelt
		\item Erzwingt fork
	\end{itemize}
\end{frame}

\begin{frame}{Dedizierter UVL Language Server}
	\begin{itemize}
		\item Kommunikation auf Websockets umstellen
		\item Monaco mit Language Server Protocol ausgestattet
		\item Multi-User UVLS
	\end{itemize}
\end{frame}

\begin{frame}{Infrastruktur und Tooling}
	\begin{mycolumns}[columns=2]
		\begin{itemize}
			\item Traefik Reverseproxy als TLS Endpoint
			\item CI / CD Pipeline
			\item Sonarqube Linter
		\end{itemize}
		\mynextcolumn
		\pic[width=0.9\textwidth]{../pics/screenshots/traefik.png}
		\pic[width=\textwidth]{../pics/screenshots/sonarqube.png}
	\end{mycolumns}
\end{frame}

\begin{frame}
	\Huge
	\centering
	\textless \texttt{/liveDemo}\textgreater

\end{frame}

\begin{frame}{Zusätzliche Features}
	\begin{mycolumns}[columns=2]
		\begin{itemize}
			\item Feature Model im Darkmode
			\item Automatisches Update von Feature Model
			\item Tutorial für Playground und UVL
			\item Upload / Download
			\item Laden von Beispielen
			\item Größenlimitierung von Feature Modellen
			\item Speichern von Feature Modellen im Browser
			\item Reconnects bei Verbindungsabbruch
		\end{itemize}
		\mynextcolumn
		\pic[width=\textwidth]{../pics/screenshots/playground.png}
	\end{mycolumns}
\end{frame}

%\section{Old Slide Layouts (Deprecated)}
%
%\begin{frame}{\insertsection}
%	\begin{note}{Note}
%		The following slide layouts are replaced by the \texttt{mycolumns}-envrionment and therefore deprecated.
%
%		Please do not use them anymore, as they are going to be removed from the template in the future.
%	\end{note}
%\end{frame}
%
%\subsection{Left and Right}
%\begin{frame}{\insertsubsection}
%    \leftandright{
%        This is an example text that is shown in the \textbf{left column}.
%    }{
%        This is an example text that is shown in the \textbf{right column}.
%    }
%	\vfill
%	\begin{note}{Explanation}
%		Both columns are visible in \textbf{handout}, \textbf{slide}, and \textbf{recording} mode (i.e., there are no animations).
%	\end{note}
%\end{frame}
%
%\subsection{Left, Middle, and Right}
%\begin{frame}{\insertsubsection}
%	\leftmiddleandright{
%		This is an example text that is shown in the \textbf{left column}.
%	}{
%		This is an example text that is shown in the \textbf{middle column}.
%	}{
%		This is an example text that is shown in the \textbf{right column}.
%	}
%	\vfill
%	\begin{note}{Explanation}
%		All columns are visible in \textbf{handout}, \textbf{slide}, and \textbf{recording} mode (i.e., there are no animations).
%	\end{note}
%\end{frame}
%
%%\recordingtrue % special recording mode for use with a greenscreen, gives you space to show yourself in a layer in front of the slides, has no effect in the handout mode
%
%\subsection{Left then Right}
%\begin{frame}{\insertsubsection}
%    \leftthenright{
%        This is an example text that is shown in the \textbf{left column}.
%    }{
%        This is an example text that is shown in the \textbf{right column}.
%    }
%	\vfill
%	\begin{note}{Explanation}
%		In \textbf{handout} mode, both columns are visible.
%
%		In \textbf{slide} and \textbf{recording} mode, only the left column is shown at the beginning, then both columns.
%	\end{note}
%\end{frame}
%
%\begin{frame}{Right then Left}
%    \rightthenleft{
%        This is an example text that is shown in the \textbf{left column}.
%    }{
%        This is an example text that is shown in the \textbf{right column}.
%    }
%	\vfill
%	\begin{note}{Explanation}
%		In \textbf{handout} mode, both columns are visible.
%
%		In \textbf{slide} and \textbf{recording} mode, only the right column is shown at the beginning, then both columns.
%	\end{note}
%\end{frame}
%
%\subsection{Left, Middle, then Right}
%\begin{frame}{\insertsubsection}
%    \leftmiddlethenright{
%        This is an example text that is shown in the \textbf{left column}.
%    }{
%        This is an example text that is shown in the \textbf{middle column}.
%    }{
%        This is an example text that is shown in the \textbf{right column}.
%    }
%	\vfill
%	\begin{note}{Explanation}
%		In \textbf{handout} mode, all columns are visible.
%
%		In \textbf{slide} and \textbf{recording} mode, only the left column is shown at the beginning, then additionally the middle column, and finally all columns.
%	\end{note}
%\end{frame}
%
%\begin{frame}{Right, Middle, then Left}
%    \rightmiddlethenleft{
%        This is an example text that is shown in the \textbf{left column}.
%    }{
%        This is an example text that is shown in the \textbf{middle column}.
%    }{
%        This is an example text that is shown in the \textbf{right column}.
%    }
%	\vfill
%	\begin{note}{Explanation}
%		In \textbf{handout} mode, all columns are visible.
%
%		In \textbf{slide} and \textbf{recording} mode, only the right column is shown at the beginning, then additionally the middle column, and finally all columns.
%	\end{note}
%\end{frame}
%
%\subsection{Left or Right}
%\begin{frame}{\insertsubsection}
%    \leftorright{
%        This is an example text that is shown in the \textbf{left column}.
%    }{
%        This is an example text that is shown in the \textbf{right column}.
%    }
%	\vfill
%	\begin{note}{Explanation}
%		In \textbf{handout mode}, both columns are visible.
%
%		In \textbf{slide mode}, only the left column is shown at the beginning and then both columns (cf. \textbf{Left then Right}).
%
%		In \textbf{recording mode}, only the left column is shown at the beginning, then an empty slide (to walk to the other side), and finally only the right column.
%	\end{note}
%\end{frame}
%
%\begin{frame}{Right or Left}
%    \rightorleft{
%        This is an example text that is shown in the \textbf{left column}.
%    }{
%        This is an example text that is shown in the \textbf{right column}.
%    }
%	\vfill
%	\begin{note}{Explanation}
%		In \textbf{handout mode}, both columns are visible.
%
%		In \textbf{slide mode}, only the right column is shown at the beginning and then both columns (cf. \textbf{Right then Left}).
%
%		In \textbf{recording mode}, only the right column is shown at the beginning, then an empty slide (to walk to the other side), and finally only the left column.
%	\end{note}
%\end{frame}
%
%\subsection{Left, Middle, or Right}
%\begin{frame}{\insertsubsection}
%    \leftmiddleorright{
%        This is an example text that is shown in the \textbf{left column}.
%    }{
%        This is an example text that is shown in the \textbf{middle column}.
%    }{
%        This is an example text that is shown in the \textbf{right column}.
%    }
%	\vfill
%	\begin{note}{Explanation}
%		In \textbf{handout mode}, all columns are visible.
%
%		In \textbf{slide mode}, only the left column is shown at the beginning, then additionally the middle column, and finally all columns (cf. \textbf{Left, Middle, then Right}).
%
%		In \textbf{recording mode}, only the left column is shown at the beginning, then only the middle column, and finally only the right column (again interleaved with empty slides).
%	\end{note}
%\end{frame}
%
%\begin{frame}{Right, Middle, or Left}
%    \rightmiddleorleft{
%        This is an example text that is shown in the \textbf{left column}.
%    }{
%        This is an example text that is shown in the \textbf{middle column}.
%    }{
%        This is an example text that is shown in the \textbf{right column}.
%    }
%	\vfill
%	\begin{note}{Explanation}
%		In \textbf{handout mode}, all columns are visible.
%
%		In \textbf{slide mode}, only the right column is shown at the beginning, then additionally the middle column, and finally all columns (cf. \textbf{Right, Middle, then Left}).
%
%		In \textbf{recording mode}, only the right column is shown at the beginning, then only the middle column, and finally only the left column (again interleaved with empty slides).
%	\end{note}
%\end{frame}

\end{document}